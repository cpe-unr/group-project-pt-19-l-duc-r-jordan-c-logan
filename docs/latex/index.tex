\label{index_md_README}%
\Hypertarget{index_md_README}%
 \mbox{\hyperlink{class_wav}{Wav}}\+: Functions\+: read\+File() -\/ read in wav file data write\+File() -\/ modify wav file data getters for 8 and 16 bit buffers and buffer\+Size destructor to delete buffer from heap


\begin{DoxyEnumerate}
\item Similar to PA4, make \mbox{\hyperlink{class_processor}{Processor}} Abstract class (aka interface).
\item \mbox{\hyperlink{class_normalization}{Normalization}} inherits to processor and implements virtual method in own class for normalizing audio
\item Noise\+Gating and \mbox{\hyperlink{class_echo}{Echo}} inherit and do similar implementations as normalization Each are own classes in separate files!! And these methods are only triggered if UI says to. {\itshape this class also helps file io with reading in wav files}
\end{DoxyEnumerate}

File IO\+: Functions\+: make\+CSVFile()


\begin{DoxyEnumerate}
\item read funtion(s) using templates for different read in type methods
\item overwrite metadata function if UI alerts to and save this change in the file
\item Probably have separate funtions in class to create and export CSV file with info that got read in at start of program
\end{DoxyEnumerate}

User Interaction/ printing\+: Functions\+: modify\+Printer() -\/ user interaction for modifying metadata processing\+Printer() -\/ user interaction for processing wav files csv\+File\+Printer() -\/ user interaction for making a CSV file


\begin{DoxyEnumerate}
\item First prompt is triggered after all the files are read in
\begin{DoxyItemize}
\item prompts user if they want to modify the metadata of any file
\begin{DoxyItemize}
\item if yes, prompt for the file and changes they want to make. Then, send to file IO classes to overwrite metadata
\item if no, move on
\end{DoxyItemize}
\end{DoxyItemize}
\item Next prompts will regard processing of certain audio files
\begin{DoxyItemize}
\item prompt user if they would like to process any audio files
\begin{DoxyItemize}
\item if yes, prompt for file and the processors they want used and send over to Wav/processor classes also prompt for file name and check if valid to be saved there
\item if no, move on
\end{DoxyItemize}
\end{DoxyItemize}
\item Last set of prompts will be for CSV file
\begin{DoxyItemize}
\item prompt user if they would like a CSV file made containing all tech. info \& metadata
\begin{DoxyItemize}
\item if yes, send to function(s) that create, write out file, and export to library
\item if no, end program!!
\end{DoxyItemize}
\end{DoxyItemize}
\end{DoxyEnumerate}

Rough overview \begin{DoxyVerb}System will read in wav files one at a time, extract all the technical info(Sample rate. etc.) and metadata from the file.
This data should be stored into an object.
The object should be added to an appropriate data structure.


Sound file data (sound buffer. etc.) should be deleted from the HEAP, not disk!!
    EX:     Wav::~Wav() {
                 if(buffer != NULL){
                    delete[] buffer;
                 }  
            }//destructor for Wav and deletes buffer from heap 



The data from each file is to be stored using a CLASS designed for the type of file 
    - MONO or STEREO: 8,16,24, or 32 bit samples

When all the file are read and processed, the user can choose to modify the metadata of any file.
        std::cout << "Do you want to modify any files metadata? 1:yes or 0:no" << std::endl;
        std::cin >> input;
        if(input == 1){    
            std::cout << "What file? " << std::endl;
            std::cin >> file;
            //change metadata//
            //save new metadata file//
        }

When the modification is complete, save the file with the new metadata.
    - The changes should override any current metadata, or create the metadata if it didn't exist before.

Then, the user may choose to process the file by choosing one or more processors(i.e. Normalization, NoiseGating, Echo).

    std::cout << "Do you want to process the modified file? 1:yes 0:no " << std::endl;
    std::cin >> input2;
    if(input2 == 1){
        std::cout << "What processing would you want done?" << std::endl;
        std::cout << "1 - Normalization" << std::endl;
        std::cout << "2 - Noise Gating" << std::endl;
        std::cout << "3 - Echo" << std::endl;
        std::cin >> input3;
        if(input3 == 1){
            //Normalization processing//
        }
        else if(input3 == 2){
            //Noise Gate processing//
        }
        else{
            //Echo processing//
        }
    }

After processing, prompt the user for a file name and check if valid:
    - if valid a new file is created with the processed audio 
    - the user is NOT allowed to save the file under a name of any of the existing files

    std::cout << "Please enter a valid filename: ";
    std::cin >> input4;
    if(input4.is_open()){

        //New file created with processed audio//

    }

Prompt user if they would like a CSV file with all the files, their technical data(sample rates, etc.) and the file metadata 
    - if no metadata exists the phrase "No metadata present" should be included in the description for that file

    std::cout << "Would you like a CSV file created containing all the files technical and metadata? 1:yes 0:no" << std::endl;
    std::cin >> input5;
    if(input5 == 1){

        //create CSV file with all the files info//

    }
\end{DoxyVerb}
 